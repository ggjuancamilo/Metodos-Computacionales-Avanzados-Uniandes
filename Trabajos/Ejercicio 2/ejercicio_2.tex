
%--------------------------------------------------------------------
%--------------------------------------------------------------------
% Formato para los talleres del curso de Métodos Computacionales
% Universidad de los Andes
%--------------------------------------------------------------------
%--------------------------------------------------------------------

\documentclass[11pt,letterpaper]{exam}
\usepackage{amsmath}
\usepackage[utf8]{inputenc}
\usepackage[spanish]{babel}
\usepackage{graphicx}
\usepackage{tabularx}
\usepackage[absolute]{textpos} % Para poner una imagen completa en la portada
\usepackage{hyperref}
\usepackage{float}

\newcommand{\base}[1]{\underline{\hspace{#1}}}
\boxedpoints
\pointname{ pt}

\extraheadheight{-0.15in}

\newcommand\upquote[1]{\textquotesingle#1\textquotesingle} % To fix straight quotes in verbatim



\begin{document}
\begin{center}
{\Large Universidad de los Andes - M\'etodos Computacionales Avanzados} \\
Ejercicio 2 - \textsc{MCMC}\\
21-04-2017\\
\end{center}



\vspace{0.3cm}


\noindent
La solución a este ejercicio debe subirse por SICUA antes de las 8:00PM
del viernes 21 de abril del 2017. 
Los c\'odigos deben encontrarse en un unico repositorio de \verb'github'
con el nombre \verb"NombreApellido_Ej2". Por ejemplo yo deber\'ia
crear un repositorio con el nombre
\verb"JaimeForero_Ej2". 

\noindent
En el repositorio estar un \'unico c\'odigo de python que
resuelve el problema propuesto.  

\vspace{0.3cm}

\begin{questions}
\question{\bf{Epicentro}}
  Una fuente s\'ismica se activo al tiempo $t=0$ en un lugar
desconocido de la Tierra. Las ondas s\'ismicas
producidas por la explosi\'on se grabaron por una red de seis
estaciones s\'ismicas ubicadas en las siguientes coordenadas (en km):
$(x_1,y_1,z_1)=(2,20,0)$,
$(x_2,y_2,z_2)=(-2,-1,0)$,
$(x_3,y_3,y_3)=(5,12,0)$,
$(x_4,y_4,z_4)=(8,10,0)$,
$(x_5,y_5,z_5)=(5,-16,0)$,
$(x_6,y_6,y_6)=(1,40,0)$. Los 
tiempos de llegada (en segundos) de las ondas s\'ismicas fueron 
$t_{obs,1}=3.23\pm\sigma_t$, $t_{obs,2}=3.82\pm\sigma_t$,
$t_{obs,3}=2.27\pm\sigma_t$, $t_{obs,4}=3.04\pm\sigma_t$,
$t_{obs,5}=5.65\pm\sigma_t$, $t_{obs,6}=6.57\pm\sigma_t$, donde
$\sigma_t=0.05$. 
\begin{itemize}
\item (60 puntos) Implemente un MCMC Hamiltoniano
  para encontrar la distribuci\'on de probabilidad de la posici\'on del
  epicentro dados los par\'ametros observacionales. Asuma que la
  velocidad de propagaci\'on de las 
  ondas es constante e igual a $5$ km/s.
  El programa debe hacer tres gr\'aficas con la distribuci\'on de
  probabilidad de $x$, $y$ y $z$ del epicentro.
\item (60 puntos) Implemente el test de Rubin-Gelman (estad\'istica
  $R^2$) para mostrar que las cadenas de Markov encontradas en el paso
  anterior efectivamente convergieron. El codigo debe preparar una
  gr\'afica acorde.
\end{itemize}

\end{questions}

\end{document}

